
\section{Problem statement}
Continuous delivery or continuous deployement (CD) is the act of releasing new software versions to the end-users as frequently as possible. This practice has been on the rise in the last decade, and with it, the need for tools to automate the deployment process. Infrastructure as code (IaC) is a practice that aims to automate the deployment of infrastructure by using code. IaC scripts are used to describe the desired state of the infrastructure, and the tools will then deploy the infrastructure to match the desired state. \\
The IaC scripts are usually stored in a version control system (VCS) such as Git. This allows the developers to collaborate on the scripts and to keep track of the changes made to the scripts. The VCS also allows the developers to review the changes made to the scripts before merging them into the main branch, just as they would do with regular code. \\
However, few other mechanisms exist to ensure the quality of the scripts. This can lead to faulty configuration being deployed to the infrastructure, which is problematic because it can lead to downtime, security breaches, and other issues. As mentionned in the original paper, in 2017, Wikimedia Commons executed a defective IaC script which led to the deletion the home directory of around 270 users. \\
The replicated paper aimed at introducing a new gating mechanism for IaC scripts by identifying the source code properties of defective scripts. Furthermore, it compares different defect prediction model which aim at identifying defective scripts before they are executed. \\


