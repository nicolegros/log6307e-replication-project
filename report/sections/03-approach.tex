
\section{Approach}
3) Methodologies for answering each RQ (How to mine the repositories, the tools employed, and ML models they've used).
\subsection{Repository mining}
\subsection{RQ1: What source code properties characterize defective infrastructure as code scripts?}
For this research question we used the reported data from the 
paper\footnote{https://figshare.com/s/ad26e370c833e8aa9712}.
We used the \emph{Mann-Whitney U} test with the Scikit Learn package
to evaluate which properties had the biggest influence on defective files. 
The null hypothesis is that the property is not different between defective and 
neutral files, and the alternative hypothesis is that the property is larger for 
defective than neutral files. As in the paper, we consider a significance level of 
95\% which means we reject the null hypothesis when $ p-value < 0.05 $. \\

We also used \emph{Cliff's Delta}\footnote{https://github.com/neilernst/cliffsDelta}
to measure how large the difference between the distribution of each characteristics
for defective and neutral files is. 

\subsection{RQ3: How can we construct defect prediction models for 
infrastructure as code scripts using the identified source code properties?}
Before using statistical learners, we completed a PCA analysis to determine 
what properties should be used. We only used the principal component that accounted
for at least 95\% of the total variance as the input for the statistical learners.
We can see in Table \ref{table:pca} that only one or two principle components 
account for 95\% of the total variance depending on the dataset. \\

With the component created, we than used it as the input for the different
statistical learners. Like the paper, we used Scikit Learn packages to construct
the models. The learners that were used are Classification Tree (CART),
K Nearest Neighbor (KNN), Logistic Regression (LR), Naive Bayer (NB) and 
Random Forest (RF). \\

To evaluate the performance of the different classification models, we used the 
same metrics as the paper (i.e. precision, recall, AUC, F-measure).
